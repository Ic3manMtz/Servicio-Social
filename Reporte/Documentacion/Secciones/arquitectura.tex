	\section{Arquitectura}
	\subsection{Diagrama de la estructura del proyecto}
	%src/
	%├── core/               # Lógica principal
	%│   ├── processors.py   # Clases de procesamiento
	%│   └── models.py       # Modelos de datos
	%├── api/                # Capa de interfaz
	%│   ├── routes.py       # Definición de endpoints
	%│   └── schemas.py      # Esquemas Pydantic
	%└── utils/              # Utilidades compartidas
	%	├── logger.py       # Configuración de logging
	%	└── helpers.py      # Funciones auxiliares

\dirtree{%
.1 Proyecto/.
.2 .venv/.
.2 src/.
.2 tests/.
.2 requirements.txt.
}
	\begin{description}
		\item[Proyecto/]  
		Raíz del proyecto que contiene todos los archivos y directorios necesarios para el desarrollo y ejecución de la aplicación.
		\item[.venv/]  
		Directorio del entorno virtual de Python, que aísla las dependencias del proyecto para evitar conflictos con otros proyectos.
		\item[src/]  
		Contiene el código fuente principal de la aplicación, donde se organizan los módulos y paquetes necesarios para su funcionamiento.
		\item[tests/]  
		Directorio dedicado a las pruebas automatizadas, que incluye tanto pruebas unitarias como de integración para asegurar la calidad del código.
		\item[requirements.txt]  
		Archivo que lista las dependencias del proyecto, especificando las versiones de los paquetes necesarios para su correcto funcionamiento.
	\end{description}

	\subsection{Descripci\'on de los componentes}

	\noindent Aquí estará la descripción de los componentes del proyecto, incluyendo sus responsabilidades y cómo interactúan entre sí. Cada componente debe ser claramente definido para facilitar la comprensión de la arquitectura general.


	\subsection{Flujo de datos}

\begin{figure}[h]
\centering
\begin{tikzpicture}[
    node distance=2cm,
    startstop/.style={rectangle, rounded corners, draw, fill=red!20},
    process/.style={rectangle, draw, fill=blue!20}
]
    \node (start) [startstop] {Inicio};
    \node (proceso) [process, below=of start] {Proceso};  % Usar "below=of" (requiere positioning)
    \draw [-Stealth] (start) -- (proceso);  % -Stealth es más moderno que ->
\end{tikzpicture}
\caption{Flujo de datos}
\end{figure}