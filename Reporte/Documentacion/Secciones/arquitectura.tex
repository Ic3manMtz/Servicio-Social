\section{Arquitectura}

\subsection{Diagrama de la estructura del proyecto}
\dirtree{%
.1 Proyecto/.
.2 .venv/.
.2 src/.
.3 database/.
.4 connection.py.
.4 creationOfModels.py.
.4 models/.
.3 features/.
.4 handler.py.
.4 video\_functions.py.
.4 video\_to\_frames\_concurrent.py.
.3 menus/.
.4 main\_menu.py.
.3 main.py.
.2 tests/.
.3 check\_requirements.py.
.2 .env.
.2 requirements.txt.
}

	\begin{description}
		\item[Proyecto/]  
		Raíz del proyecto que contiene todos los archivos y directorios necesarios para el desarrollo y ejecución de la aplicación.
		\item[.venv/]  
		Directorio del entorno virtual de Python, que aísla las dependencias del proyecto para evitar conflictos con otros proyectos.
		\item[src/]  
		Contiene el código fuente principal de la aplicación, donde se organizan los módulos y paquetes necesarios para su funcionamiento.
		\item[tests/]  
		Directorio dedicado a las pruebas automatizadas, que incluye tanto pruebas unitarias como de integración para asegurar la calidad del código.
		\item[requirements.txt]  
		Archivo que lista las dependencias del proyecto, especificando las versiones de los paquetes necesarios para su correcto funcionamiento.
	\end{description}

\subsection{Descripci\'on de los componentes}

\begin{itemize}
    \item \textbf{database}: Gestiona la conexión y manipulación de la base de datos. Incluye archivos para la conexión (\texttt{connection.py}), la creación de modelos (\texttt{creationOfModels.py}) y un subdirectorio \texttt{models} donde se definen las estructuras de datos.
    \item \textbf{features}: Contiene la lógica principal de las funcionalidades de la aplicación, como el procesamiento de video, análisis de frames y reconstrucción.
    \item \textbf{menus}: Incluye los módulos relacionados con la interfaz de usuario y la navegación dentro de la aplicación. El subdirectorio \texttt{shared} almacena componentes reutilizables entre diferentes menús.
\end{itemize}


\subsection{Flujo de datos}
\begin{figure}[h]
\centering
\begin{tikzpicture}[
    node distance=0.7cm,  % Espaciado más ajustado
    startstop/.style={
        rectangle, 
        rounded corners, 
        draw, 
        fill=red!20,
        minimum width=3cm,
        text width=2.8cm,
        align=center
    },
    process/.style={
        rectangle, 
        draw, 
        fill=blue!20,
        minimum width=3cm,
        text width=2.8cm,
        align=center
    }
]
    \node (start) [startstop] {Grabación de video};
    \node (convertion) [process, below=of start] {Conversión del video a frames};
    \node (analysis) [process, below=of convertion] {Análisis de frames}; 
	\node (reconstruction) [process, below=of analysis] {Reconstrucción de video};
	\node (end) [startstop, below=of reconstruction] {Análisis de los datos obtenidos};
    
    % Conexiones con flechas
    \draw [-Stealth, thick] (start) -- (convertion);
    \draw [-Stealth, thick] (convertion) -- (analysis);
	\draw [-Stealth, thick] (analysis) -- (reconstruction);
	\draw [-Stealth, thick] (reconstruction) -- (end);
\end{tikzpicture}
\caption{Flujo de datos}
\end{figure}