	\section{Introducci\'on}
	\subsection{Descripci\'on general del proyecto}
	\noindent La simulaci\'on de una red de comunicaciones en donde intervienen dispositivos personales de comunucaci\'on requiere contar con modelos que representen fielmente los patrones de movimiento de las personas. De otra manera, la utilidad de las conclusiones que se puedan obtener de esa simulaci\'on es limitada. 

	\noindent Para avanzar hacia la definici\'on de un modelo de movilidad humana grupal, se propone la construcci\'on de una base de datos de videos a\'ereos (capturados por un dron) y su an\'alisis v\'ia herramientas de IA. Esto nos permitir\'a determinar algunas caracter\'isticas de la movilidad de inter\'es. 
	
	\subsection{Objetivos y prop\'ositos}
	\noindent El objetivo principal del proyecto es contar con una caracterizaci\'on de los grupos de humanos que se desplazan juntos. Identificando las caracter\'isticas estad\'isticas de los grupos de humanos que se desplazan juntos.

	\noindent Los prop\'ositos del proyecto son:
	\begin{itemize}
		\item Construir una base de datos de videos a\'ereos de grupos humanos.
		\item Usar un modelo de IA que permita identificar y caracterizar los grupos humanos en los videos.
		\item Analizar los patrones de movimiento y las interacciones entre los grupos humanos.
	\end{itemize}

	\subsection{Alcance del sistema}
	\noindent El sistema se enfoca en la recolecci\'on de datos a\'ereos de grupos humanos y su an\'alisis utilizando herramientas de IA. El alcance incluye:
	\begin{itemize}
		\item Captura de videos a\'ereos mediante un dron en diferentes escenarios y condiciones.
		\item Almacenamiento y organizaci\'on de los videos en una base de datos estructurada.
		\item Procesamiento de los videos para la detecci\'on y seguimiento de individuos y grupos humanos.
		\item Extracci\'on de caracter\'isticas relevantes sobre la movilidad y las interacciones grupales.
		\item Generaci\'on de reportes y visualizaciones de los resultados obtenidos.
	\end{itemize}
	\noindent No se considera dentro del alcance el desarrollo de hardware de drones ni la implementaci\'on de modelos de IA desde cero; se utilizar\'an herramientas y modelos existentes.

	\newpage