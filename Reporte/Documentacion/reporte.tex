%%%%%%   TIPO DE DOCUMENTO: Artículo   %%%%%%
\documentclass[letterpaper,11pt,twoside]{report}

\usepackage[spanish]{babel} %Idioma
\usepackage{graphicx} %Imágenes
\usepackage[utf8]{inputenc} %Acentos
\usepackage{hyperref} %Links
\usepackage{xspace}

\usepackage{listings} % Código fuente
\usepackage{dirtree} % Árboles para mostrar directorios
\usepackage[utf8]{inputenc} % Codificación UTF-8 para acentos sencillos

\usepackage{tikz}
\usetikzlibrary{shapes, arrows}
\usetikzlibrary{arrows.meta}  % ¡Necesario para las flechas!
\usetikzlibrary{positioning}  % Para mejor control de nodos

\usepackage{verbatim}
\usepackage{amsmath, amssymb}
\usepackage{amsmath}
\usepackage[active]{srcltx}
\usepackage{amssymb}
\usepackage{amscd}
\usepackage{makeidx}
\usepackage{amsthm}
\usepackage{algpseudocode}
\usepackage{algorithm}
\usepackage{float}
\usepackage{caption}

\renewcommand{\baselinestretch}{1}
\renewcommand{\thesection}{\arabic{section}}

\setcounter{page}{1}
\setlength{\textheight}{21.6cm}
\setlength{\textwidth}{14cm}
\setlength{\oddsidemargin}{1cm}
\setlength{\evensidemargin}{1cm}
\pagestyle{myheadings}
\captionsetup[figure]{position=below, skip=0pt}
\thispagestyle{empty}
\markboth{\small{Servicio Social, Jorge Mart\'inez.}}{\small{.}}
\date{}

\begin{document}
    \centerline{\bf Servicio Social, Trimestre: 25-I, 2025}
    \centerline{}
    \centerline{}
    \begin{center}
    \Large{\textsc{Construcci\'on de una base de datos de videos a\'ereos y su an\'alisis v\'ia herramientas
de IA
}}
    \end{center}
    \centerline{}
    \centerline{\bf {Martínez Buenrostro Jorge Rafael.}}
    \centerline{}
    \centerline{Universidad Aut\'onoma Metropolitana}
    \centerline{Unidad Iztapalapa, M\'exico}
    \centerline{$molap96@gmail.com$}
    \newtheorem{Theorem}{\quad Theorem}[section]
    \newtheorem{Definition}[Theorem]{\quad Definition}
    \newtheorem{Corollary}[Theorem]{\quad Corollary}
    \newtheorem{Lemma}[Theorem]{\quad Lemma}
    \newtheorem{Example}[Theorem]{\quad Example}
    \bigskip
    \textbf{Resumen:}  Este documento describe el proyecto cuaya meta es contar con una caracterización de los grupos de humanos que se desplazan juntos. Identificando las características estadísticas de los grupos de humanos que se desplazan juntos.

		\section{Introducci\'on}
	\subsection{Descripci\'on general del proyecto}
	\noindent La simulaci\'on de una red de comunicaciones en donde intervienen dispositivos personales de comunucaci\'on requiere contar con modelos que representen fielmente los patrones de movimiento de las personas. De otra manera, la utilidad de las conclusiones que se puedan obtener de esa simulaci\'on es limitada. 

	\noindent Para avanzar hacia la definici\'on de un modelo de movilidad humana grupal, se propone la construcci\'on de una base de datos de videos a\'ereos (capturados por un dron) y su an\'alisis v\'ia herramientas de IA. Esto nos permitir\'a determinar algunas caracter\'isticas de la movilidad de inter\'es. 
	
	\subsection{Objetivos y prop\'ositos}
	\noindent El objetivo principal del proyecto es contar con una caracterizaci\'on de los grupos de humanos que se desplazan juntos. Identificando las caracter\'isticas estad\'isticas de los grupos de humanos que se desplazan juntos.

	\noindent Los prop\'ositos del proyecto son:
	\begin{itemize}
		\item Construir una base de datos de videos a\'ereos de grupos humanos.
		\item Usar un modelo de IA que permita identificar y caracterizar los grupos humanos en los videos.
		\item Analizar los patrones de movimiento y las interacciones entre los grupos humanos.
	\end{itemize}

	\subsection{Alcance del sistema}
	\noindent El sistema se enfoca en la recolecci\'on de datos a\'ereos de grupos humanos y su an\'alisis utilizando herramientas de IA. El alcance incluye:
	\begin{itemize}
		\item Captura de videos a\'ereos mediante un dron en diferentes escenarios y condiciones.
		\item Almacenamiento y organizaci\'on de los videos en una base de datos estructurada.
		\item Procesamiento de los videos para la detecci\'on y seguimiento de individuos y grupos humanos.
		\item Extracci\'on de caracter\'isticas relevantes sobre la movilidad y las interacciones grupales.
		\item Generaci\'on de reportes y visualizaciones de los resultados obtenidos.
	\end{itemize}
	\noindent No se considera dentro del alcance el desarrollo de hardware de drones ni la implementaci\'on de modelos de IA desde cero; se utilizar\'an herramientas y modelos existentes.

	\newpage

		\section{Requisitos}
	\subsection{Requisitos del sistema}
	\begin{itemize}
		\item Base de datos PostgreSQL 17.5.1
		\item Versión mínima de Python 3.13.3
		\item Dependencias principales:
		\begin{itemize}
			\item torch
			\item torchvision
			\item numpy
			\item pandas
			\item scikit-learn
			\item matplotlib
			\item seaborn
		\end{itemize}
		\item Dependencias procesamiento de imágenes/video:
		\begin{itemize}
			\item ultralytics
		\end{itemize}
		\item Dependencias de tracking:
		\begin{itemize}
			\item deep-sort-realtime
		\end{itemize}
		\item Dependencias de multimedia:
		\begin{itemize}
			\item ffmpeg-python
		\end{itemize}
		\item Dependencias de base de datos:
		\begin{itemize}
			\item sqlalchemy
			\item psycopg2-binary
		\end{itemize}
		\item Dependencias de variables de entorno:
		\begin{itemize}
			\item python-dotenv
		\end{itemize}
		\item Dependencias de barras de progreso:
		\begin{itemize}
			\item tqdm
		\end{itemize}
	
	\end{itemize}


	\subsection{Instrucciones de instalaci\'on}
	\begin{enumerate}
		\item Instalar PostgreSQL 17.5.1 desde la \href{https://www.postgresql.org/download/}{p\'agina oficial} de PostgreSQL.
		\item Clonar el repositorio del proyecto desde \href{https://github.com/Ic3manMtz/Servicio-Social.git}{Github} el proyecto se encuentra dentro de la carpeta \texttt{Implementaci\'on}.
		\item Crear un entorno virtual en Python:
		\begin{lstlisting}[language=bash]
python -m venv .venv
source .venv/bin/activate  # Linux/Mac
.\.venv\Scripts\activate   # Windows
		\end{lstlisting}
		\item Instalar las dependencias del proyecto:
		\begin{lstlisting}[language=python]
pip install -r requirements.txt
		\end{lstlisting}
		\item Configurar variable de entorno:
		\begin{lstlisting}
cp .env.example .env
nano .env  # Editar valores
		\end{lstlisting}
		\item Verificar instalaci\'on:
		\begin{lstlisting}[language=python]
python tests/check_requirements/
		\end{lstlisting}
	\end{enumerate}

\noindent Una vez descargadas las dependencias necesarias. Si es la primera vez que se usará el programa el siguiente paso es crear las tablas de la base de datos utilizando los modelos ya creados.
\begin{enumerate}
	\item Se tiene que abrir el manejar de la base de datos, en el caso de PostgreSQL se llama pgAdmin, para este proyecto se recomienda usar la versión cuatro.
	\item Una vez abierto se tiene que crear una base de datos llamada \texttt{VideoData}. El nombre de la base de datos, el usuario y la contraseña se pueden modificar en la variable de entorno \texttt{DB\_CONNECTION\_STRING}, se encuentra dentro del archivo \texttt{.env}.
	\item Con la base de datos creada lo siguiente es ejecutar el archivo \texttt{creationOfModels.py}, se encuentra dentro del paquete \texttt{database} del proyecto.
	\item Al terminar la ejecución ya deben estar las mismas tablas que modelos dentro de la base de datos.
\end{enumerate}


	\section{Arquitectura}

\subsection{Diagrama de la estructura del proyecto}

\begin{figure}[h]

\dirtree{%
.1 Proyecto/.
.2 .venv/.
.2 src/.
.3 converted\_videos/.
.3 database/.
.4 connection.py.
.4 creationOfModels.py.
.4 db\_crud.py.
.4 models/.
.3 features/.
.4 detect\_tracking.py.
.4 handler.py.
.4 reconstruct\_video.py.
.4 video\_functions.py.
.4 video\_to\_frames\_concurrent.py.
.3 frames/.
.3 menus/.
.4 main\_menu.py.
.3 main.py.
.3 yolo8n.pt.
.2 tests/.
.3 check\_requirements.py.
.2 .env.
.2 requirements.txt.
}

    \caption{Diagrama de la estructura del proyecto}
    \label{project_struct}
\end{figure}

\subsection{Descripci\'on de los componentes}

\noindent El orden en el que se presentan los componentes será de acuerdo al diagrama de la \textit{Figura.} \ref{project_struct}

\subsubsection*{src}
\noindent Contiene el código fuente principal de la aplicación, donde se organizan los módulos y paquetes necesarios para su funcionamiento.

\paragraph{database}
\noindent Este paquete contiene todo lo necesario para la concexión con la base de datos. Así como la creación de las tablas de acuerdo al modelo establecido
\begin{description}
    \item[connection.py] - Script que carga las variables de entorno para poder obtener el \texttt{connection string} para realizar la conexión con la base de datos.
    
    \item[creationOfModels.py] - Este script crea una conexión con la base de datos para poder crear las tablas con base en los modelos creados.
    
    \item[db\_crud.py] - Script que contiene los métodos CRUD para la base de datos.
    
    \item[models.py] - Script que contiene la descripción de los modelos usados en el proyecto. Estos modelos se usan para la creación de las tablas de la base de datos
\end{description}


\paragraph{features}
\noindent Este paquete contiene los scripts que manipulan los videos.
\begin{description}
    \item[handler.py] - Este archivo es una clase de Python. Tiene dos atributos: el primero guarda la ruta en la que se guardan los videos que se analizarán; mientras que la segunda guarda la ruta en la que se guardarán los resultados que se generen durante la ejecución del programa. \\La clase contiene varios métodos. El primero se llama \texttt{configure\_requirements}, su función es instalar las dependecias necesarias para el funcionamiento del programa. También contiene métodos para obtener y asignar los valores de las rutas antes mencionada. \\Los demás métodos manejan la respuesta de los menús para poder direccionar al usuario de forma correcta.
    
    \item[video\_functions.py] - Esta clase contiene métodos abstractos que llaman a otros scripts de Python para poder realizar el manejo, análisis y creación de videos.
    
    \item[video\_to\_frames\_concurrent.py] - Script que con base en un directorio convierte todos los videos \texttt{.mp4} a frames de forma concurrente. El script crea una carpeta por cada video dentro del directorio, tiene el mismo nombre que el video convertido.\\ La variable \texttt{sampling\_rate} puede ser ajustada de acuerdo a la necesidad. Al empezar el análisis de un video se guardan sus metadatos en la tabla \texttt{VideoMetadata} de la base de datos.
\end{description}

\paragraph{frames}
\noindent Esta carpeta contiene carpetas que contienen los frames de cada video convertido. Cada carpeta tiene el mismo nombre que el video del cual se generaron los frames. Dentro de cada carpeta se guardan los frames en formato \texttt{.jpg}.

\paragraph{menus}
\noindent Este paquete contiene los scripts que crean los menus y mensajes que se ven en la terminal.
\paragraph{main.py}
\noindent Este script contiene los menus que se ven en la terminal así como algunos de los mensajes que se crean durante la interacción con el usuario.

\subsubsection*{tests}
\noindent Directorio dedicado a las pruebas automatizadas, que incluye tanto pruebas unitarias como de integración para asegurar la calidad del código.
\paragraph{check\_requirements.py}
\noindent Este script revisa las dependecias contenidas en el archivo \texttt{requirements.txt} para poder determinar que dependencias hacen falta en el sistema. Antes de la instalación de las dependencias faltantes se actualiza el \texttt{pip} para evitar errores.

\subsubsection*{\. env}
\noindent Este archivo contiene las variables de entorno necesarias para el programa. 

\subsubsection*{requirements.txt}
\noindent Archivo que lista las dependencias del proyecto, especificando las versiones de los paquetes necesarios para su correcto funcionamiento.

\subsection{Flujo de datos}
\begin{figure}[h]
\centering
\begin{tikzpicture}[
    node distance=0.7cm,  % Espaciado más ajustado
    startstop/.style={
        rectangle, 
        rounded corners, 
        draw, 
        fill=red!20,
        minimum width=3cm,
        text width=2.8cm,
        align=center
    },
    process/.style={
        rectangle, 
        draw, 
        fill=blue!20,
        minimum width=3cm,
        text width=2.8cm,
        align=center
    }
]
    \node (start) [startstop] {Grabación de video};
    \node (convertion) [process, below=of start] {Conversión del video a frames};
    \node (analysis) [process, below=of convertion] {Análisis de frames}; 
	\node (reconstruction) [process, below=of analysis] {Reconstrucción de video};
	\node (end) [startstop, below=of reconstruction] {Análisis de los datos obtenidos};
    
    % Conexiones con flechas
    \draw [-Stealth, thick] (start) -- (convertion);
    \draw [-Stealth, thick] (convertion) -- (analysis);
	\draw [-Stealth, thick] (analysis) -- (reconstruction);
	\draw [-Stealth, thick] (reconstruction) -- (end);
\end{tikzpicture}
\caption{Flujo de datos}
\end{figure}

		\section{Configuraci\'on}
	\subsection{Archivos de configuraci\'on}
	\subsection{Par\'ametros ajustables}
\end{document}

